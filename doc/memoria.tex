\documentclass[a4paper, 12pt]{article}

\usepackage[utf8]{inputenc}
\usepackage[spanish]{babel}
\usepackage[T1]{fontenc}
\usepackage{tikz}
\usepackage{fancyhdr}
\usepackage{graphicx}
\usepackage{minted}
\usepackage[a4paper, total={6.8in, 8in}]{geometry}
\usetikzlibrary{positioning}

\pagestyle{fancy}
\fancyhf{}
\rhead{Martín Romera Sobrado}
\chead{FIA}
\lhead{PEC2}
\cfoot{\thepage}

\setlength{\headheight}{52pt}

\begin{document}

    \begin{titlepage}
        
        \centering
		\vspace*{1cm}
		
		\Huge
		\textbf{Fundamentos de Inteligencia Artificial}\\
		\textbf{PEC2 - 2021}
		
		\vspace{0.5cm}
		\LARGE
		Centro Asociado de la UNED en Bizkaia\\
		\vspace{1.5cm}
		
		\textbf{Martín Romera Sobrado}\\
		\textbf{Bilbao}
		
		\vfill
		
		\vspace{0.8cm}
		\Large
		\today

    \end{titlepage}

	\tableofcontents
	\newpage

	\section{Introducción}

		En este proyecto se presenta \textit{El Pactometro 4M Automático} basado 
	en la herramienta de la cadena de televisión de \textit{La Sexta} para 
	calcular posibles pactos entre partidos políticos tras unas elecciones. Sin 
	embargo la herramienta original carece de una caracteristica bastante 
	importante, que es comentar la validez del pacto calculado.\\\mbox{}

		Valorar un pacto puede parecer algo muy humano, ya que un pacto es 
	primitivamente una relación humana. Sin embargo, un pacto político también
	se basa en una serie de \textbf{reglas} lógicas. Por ejemplo, es más dificil
	que un partido de izquierdas pacte un partido de derechas, o es dificil que 
	pacten dos partidos que han asegurado ante los medios que quieren pactar 
	entre ellos o que solo uno de ellos quiera pactar. \textit{El Pactometro 4M 
	Automático} se aprovecha de este tipo de lógica para convertirse en un 
	\textit{Sistema Experto para la valoración de pactos tras las elecciones del
	4 de Mayo en la Comunidad de Madrid}.\\\mbox{}

	\section{Modelado del conocimiento del dominio}

		El dominio sobre el que trabajará el sistema como ya se ha mencionado 
	antes son \textbf{los pactos de las elecciones del 4M}. Para modelar este 
	conocimiento nos aprovechamos de la información que han proporcionado los
	partidos a los medios de comunicación, como la intención de pacto de cada
	partido, o el concepto que tiene la sociedad de los partidos, como el 
	espectro político de cada uno de estos (izquierda o derecha) y en que 
	posición se encuentran en él (más extremos o más centristas).

		Cabe decir que se ha simplificado el método de cálculo de pactos, de 
	forma que no tiene en cuenta abstenciones, solo apoyos. Si un conjunto de
	partidos suma una mayoría absoluta (69 escaños o más) el sistema considerará
	que el pacto se podrá evaluar y considerará al lider del partido con más 
	escaños futuro hipotético presidente.\\\mbox{}

	\section{Descripción del sistema}

		Para comenzar a introducir pactos se deberá introducir el comando 
	\texttt{[main].}, para cargar los modulos, seguido de \texttt{init.}, donde 
	el usuario podrá introducir al sistema los resultados sobre los que quiere 
	trabajar. 
	\begin{figure}[ht!]
		\begin{minted}[breaklines]{text}
?- [main].
true.

?- init.
Introduce el número de escaños que ha obtenido Unidas Podemos
|: 10.
Introduce el número de escaños que ha obtenido Más Madrid
|: 24.
Introduce el número de escaños que ha obtenido PSOE
|: 24.
Introduce el número de escaños que ha obtenido Ciudadanos
|: 0.
Introduce el número de escaños que ha obtenido Partido Popular
|: 65.
Introduce el número de escaños que ha obtenido VOX
|: 13.
el Partido Popular con 65 escaños, se sitúa como primera fuerza parlamentaria.
Para comenzar a calcular pactos, lanza el comando "pactometro."
true.
		\end{minted}
		\caption{Inicialización del sistema, con los resultados electorales del
		4M}
		\label{f:init}
	\end{figure}

	El sistema tras introducir los datos nos dice quien habría ganado las 
elecciones por numero de escaños. La frase con la que dice esto se elige según 
el número de escaños que haya obtenido ese partido (para quitarle monotoneidad
al sistema).\\\mbox{}

	\begin{figure}[ht!]
		\inputminted[firstline=29,lastline=55,breaklines]
		{prolog}{../src/comentarios.pl}
		\caption{Elección de frase aleartoria}
		\label{f:comentario_ganador}
	\end{figure}

	Tras esto podemos introducir el comando \texttt{pactometro.} para empezar a 
calcular pactos. Este comando nos pedirá primero que numero de partidos va a 
formar el pacto y despues introducir los partidos que formarian ese pacto
\footnote{Ojo, el pactometro no tiene en cuenta el concepto de abstención, las 
abstenciones se tienen que establecer como posiciones a favor o en contra
dependiendo de si esa abstención habilitaría o deshabilitaría un pacto}.
\\\mbox{}

	Pongamos por ejemplo de pacto más posible despues de las elecciones, una 
coalción de las dos fuerza de derechas con representación en el parlamento
, uno entre el \textit{Partido Popular} y \textit{VOX}. La figura 
\ref{f:pactometro_1} muestra el resultado de la ejecución del pactometro para
ese posible pacto.\\\mbox{}

\begin{figure}[ht!]
	\begin{minted}[breaklines]{text}
?- pactometro.
Cuantos partidos va a formar el pacto: 
|: 2.
2 restantes.
Introduce un partido de usando el siguiente código: 
up : Unidas Podemos
mm : Más Madrid
psoe : PSOE
cs : Ciudadanos
pp : Partido Popular
vox : VOX
|: pp.
1 restantes.
Introduce un partido de usando el siguiente código: 
up : Unidas Podemos
mm : Más Madrid
psoe : PSOE
cs : Ciudadanos
pp : Partido Popular
vox : VOX
|: vox.
Una coalición entre Partido Popular, y VOX daría paso a un gobierno de derechas en la Asamblea de Madrid liderado por Isabel Díaz Ayuso con un respaldo de 78 escaños.
true .
	\end{minted}
	\caption{Resultado del pactometro entre los dos partidos de derechas}
	\label{f:pactometro_1}
\end{figure}

	Obviamente este no es el único pacto que forma mayoría. Digamos que todos
los partidos se alian en contra del \textit{Partido Popular}. La respuesta del 
\textit{Pactometro} sería algo mal educada pero acorde a lo que le hemos pedido 
que nos valore, como podemos ver en la figura \ref{f:pactometro_2}.\\\mbox{}

\begin{figure}[ht!]
	\begin{minted}{text}
Tú fumas crack
true .
	\end{minted}
	\caption{Resultado del pactometro para un pacto muy improbable 
	(\textit{Unidas Podemos, Más Madrid, PSOE} y \textit{VOX})}
	\label{f:pactometro_2}
\end{figure}

	El \textit{Pactometro} esta diseñado para responder con una educación
proporcional al sentido que tenga el pacto que se le proponga. \\\mbox{}

	Pongamos un ejemplo más, en este caso vamos a introducir unos datos de un 
sondeo que hizo \textit{El Plural}, que daría la victoria al bloque de 
izquierdas con los resultados del cuadro \ref{f:encuesta_plural}.\\\mbox{}

\begin{table}[ht!]
	\centering
	\begin{tabular}{|c|c|}
		\hline
		\textit{Unidas Podemos} & 10 \\\hline
		\textit{Más Madrid}		& 26 \\\hline
		\textit{PSOE}			& 33 \\\hline
		\textit{Ciudadanos}		& 0	 \\\hline
		\textit{PP}				& 57 \\\hline
		\textit{VOX}			& 10 \\\hline
	\end{tabular}
	\caption{Sondeo de El Plural}
	\label{f:encuesta_plural}
\end{table}

	El pacto con más posibilidades sería el de los 3 partidos del bloque de 
izquierdas (\textit{Unidas Podemos, Más Madrid} y \textit{PSOE}). Si le 
planteamos este pacto al \textit{Pactometro} nos dará como resultado la 
frase de la figura \ref{f:pactometro_3}\\\mbox{}

\begin{figure}[ht!]
	\begin{minted}[breaklines]{text}
Un pacto entre Unidas Podemos, Más Madrid, y PSOE podrían dar paso a un gobierno progresista liderado por Ángel Gabilondo con 69 escaños de respaldo. Sin embargo Ángel Gabilondo ha advertido que no tenía intención de pactar con Pablo Iglesias lo que podría bloquear esta posibilidad.
true .
	\end{minted}
	\caption{Respuesta a los resultados del sondeo con un pacto del bloque de
	izquierdas}
	\label{f:pactometro_3}
\end{figure}

	Como el pacto es menos posible ya que el candidato del \textit{PSOE} dijo, 
que no tenía intención de pactar con \textit{Unidas Podemos}, el 
\textit{Pactometro} nos lo hace saber.

Para calcular el sentdo del pacto, este utiliza los siguientes mecanismos:

\begin{itemize}
	\item Comprobar que se encuentran en el mismo espectro.
	\item Comprobar si quieren pactar entre ellos.
	\item Calcular la mayor distancia dentro del espectro político.
\end{itemize}

	Todo este calculo lo realiza el modulo \texttt{pactos.pl}. Veamos paso a 
paso como desarrolla este calculo el sistema. Una vez introducido los partidos 
que van a formar el pacto se llama a la regla que se muestra en la figura 
\ref{f:valorar_pactos}.\\\mbox{}

\begin{figure}[ht!]
	\inputminted[breaklines,firstline=47,lastline=81]{prolog}{../src/pactos.pl}
	\caption{Regla para la valoración del pacto}
	\label{f:valorar_pactos}
\end{figure}

	Esta regla primero elimina los partidos que no han obtenido escaños y los 
que se encuentran repetidos en la lista. Tras eso comprueba si el conjunto de 
partidos tiene mayoría o no. Después llega a una estructura similar a la que en
programación imperativa podría ser un \textit{switch/case} en el que comprueba 
primero si solo hay un partido con mayoría en la coalición, si no comprueba si 
los partidos quieren pactar entre ellos en ambas direcciones, si no comprueba si 
al menos una de las partes del pacto querría participar en el pacto (como podría
ocurrir en un pacto \textit{PSOE} con \textit{Unidas Podemos})
y finalmente si hay si hay partidos dentro del hipotético pacto que no quiere 
pactar entre ellos. Cada una de estas situaciones llevaría a una regla de 
comentario que generaría un comentario acorde al pacto. Para realizar estas 
evaluaciones se aprovecha de unas reglas auxiliares definidas en la figura 
\ref{f:auxiliares_eval}.\\\mbox{}

\begin{figure}[ht!]
	\inputminted[breaklines,firstline=86,lastline=87]{prolog}
	{../src/pactos.pl}
	\inputminted[breaklines,firstline=91,lastline=105]{prolog}
	{../src/pactos.pl}
	\caption{Reglas auxiliares}
	\label{f:auxiliares_eval}
\end{figure}

	Todos estos comentarios se encuentran dentro del modulo 
\texttt{comentarios.pl}. Para evaluar mayorías, numero de escaños y posición en 
el espectro político se utilizan las reglas y constantes del modulo 
\texttt{partidos.pl}. Este modulo también define algunas constantes interesantes
para la generación de comentarios.

\section{Complicaciones durante el desarrollo}

	La principal dificultad del desarrollo de este proyecto ha sido la 
utlización de un nuevo lenguaje de un paradigma al que no estoy acostumbrado con
poco tiempo de margen (culpa mia por haber empezado tarde). Es por eso que hay 
algunas cosas dentro del código que igual podrían mejorarse. La optimización del 
código igual ha sido lo más dificil, todavía no me manejo del todo bien con la 
orden de corte y la he utilizado basicamente para que no entrara en bucles 
tontos como en los algoritmos de ordenación, en la estructura tipo 
\texttt{switch/case} de la valoración del pacto y poco más. Puede que se pudiera 
haber optmizado mejor de otra forma, pero tal vez necesite más práctica con el
lenguaje para verlo con mayor claridad.\\\mbox{}

	La implementación modular que he utilizado al principio del desarrollo 
también ha sido un dolor de cabeza. Es por eso que hay unas pocas reglas donde
no deberían estar. Sin embargo al final me he hecho más o menos al sistema y he 
podido finalizar el proyecto de forma modular para que sea más fácil leerlo.
\\\mbox{}

	Como he dicho antes el tiempo ha sido otro problema que ha hecho que algunas
características se hayan quedado fuera del proyecto final. Aquí listo algunas de
ellas:
\begin{itemize}
	\item Poder marcar abstenciones.
	\item Comentarios más precisos para la evaluación de pactos concretos.
	\item Mayor sensación de aleatoriedad con las respuestas.
	\item Evaluación de votos. (Algo que afecta a estos resultados de hecho ya
		  que el sistema reconoce que \textit{PSOE} y \textit{Más Madrid} tienen
		  el mismo apoyo cuando \textit{Más Madrid} tiene más votos).
	\item Comparación de los resultados con los de las pasadas elecciones.
	\item Generación automática del pacto más posible.
\end{itemize}
De todas formas, a pesar de las limitaciones el sistema logra dar unos
resultados con sentido.\\\mbox{}

	Otra complejidad del desarrollo aunque no ligada al propio sistema si no más 
al tema, es que es un tema ``sensible''. Son unas elecciones que acaban de 
ocurrir, y que han estado muy polarizadas y han tenido una participación masiva.
Añadirle un toque de humor a un tema como este no ha sido fácil, ya que no 
quería mostrar ni mis ideas políticas ni ofender a nadie. Espero haber cumplido 
con esto, y si no ha sido así pido disculpas por adelantado. 

\newpage

\section*{Listados del código}

\subsection*{main.pl}
	\inputminted[]{prolog}{../src/main.pl}
\subsection*{partidos.pl}
	\inputminted[]{prolog}{../src/partidos.pl}
\subsection*{pactos.pl}
	\inputminted[]{prolog}{../src/pactos.pl}
\subsection*{util.pl}
	\inputminted[]{prolog}{../src/util.pl}
\subsection*{comentarios.pl}
	\inputminted[]{prolog}{../src/comentarios.pl}
\newpage


\end{document}